%%%%%%%%%%%%%%%%%%%%%%%%%%%%%%%%%%%%%%%%%%%%%%%%%%%%%%%%%%%%%%%%%%%%%%%%%%%%%%%%
\section{GitLab vs GitHub CI}
{   
	\usebackgroundtemplate{
		\vbox to \paperheight{\vfil\hbox to \paperwidth{\hfil\includegraphics[height=\paperheight]{pipe.png}\hfil}\vfil}
		%https://www.flaticon.com/free-icon/decoration_2788716
		%<a href="https://www.flaticon.com/free-icons/decoration" title="decoration icons">Decoration icons created by Freepik - Flaticon</a>
		
	}
	\frame{
		\frametitle{CI Platforms Comparison}
		\begin{mdframed}[tikzsetting={draw=white,fill=white,fill opacity=0.8,
				line width=0pt},backgroundcolor=none,leftmargin=0,
			rightmargin=150,innertopmargin=4pt,roundcorner=10pt]
			\tableofcontents[currentsection,sections={1-6},hideothersubsections]
		\end{mdframed}
	}
}

%%%%%%%%%%%%%%%%%%%%%%%%%%%%%%%%%%%%%%%%%%%%%%%%%%%%%%%%%%%%%%%%%%%%%%%%%%%%%%%%
\begin{frame}[fragile]
	\frametitle{Configuration Files}
	\begin{columns}
		\begin{column}{0.5\textwidth}
			\textbf{GitHub Actions}
			\begin{itemize}
				\item \texttt{.github/workflows/*.yml}
				\item Multiple workflow files
				\item Each file = one workflow
			\end{itemize}
			\begin{lstlisting}[language=yaml, style=Shell]
name: CI
on: [push, pull_request]
jobs:
  test:
    runs-on: ubuntu-latest
    steps: [...]
			\end{lstlisting}
		\end{column}
		\begin{column}{0.5\textwidth}
			\textbf{GitLab CI}
			\begin{itemize}
				\item Single \texttt{.gitlab-ci.yml}
				\item All jobs in one file
				\item Stages organize jobs
			\end{itemize}
			\begin{lstlisting}[language=yaml, style=Shell]
stages:
  - test
  - deploy

test-job:
  stage: test
  script: [...]
			\end{lstlisting}
		\end{column}
	\end{columns}
\end{frame}

%%%%%%%%%%%%%%%%%%%%%%%%%%%%%%%%%%%%%%%%%%%%%%%%%%%%%%%%%%%%%%%%%%%%%%%%%%%%%%%%
\begin{frame}
	\frametitle{Key Differences}
	\begin{columns}
		\begin{column}{0.5\textwidth}
			\textbf{GitHub Actions}
			\begin{itemize}
				\item \textbf{Runners:} GitHub-hosted or self-hosted
				\item \textbf{Marketplace:} Extensive action ecosystem
				\item \textbf{Triggers:} Rich event system
				\item \textbf{Matrix builds:} Built-in support
				\item \textbf{Secrets:} Repository/organization level
			\end{itemize}
		\end{column}
		\begin{column}{0.5\textwidth}
			\textbf{GitLab CI}
			\begin{itemize}
				\item \textbf{Runners:} Shared, group, or project runners
				\item \textbf{Images:} Docker-centric approach
				\item \textbf{Triggers:} Branch/tag based mainly
				\item \textbf{Parallel jobs:} Manual configuration
				\item \textbf{Variables:} Project/group/instance level
			\end{itemize}
		\end{column}
	\end{columns}
\end{frame}

%%%%%%%%%%%%%%%%%%%%%%%%%%%%%%%%%%%%%%%%%%%%%%%%%%%%%%%%%%%%%%%%%%%%%%%%%%%%%%%%
\begin{frame}[fragile]
	\frametitle{Syntax Comparison: Python Testing}
	\begin{columns}
		\begin{column}{0.5\textwidth}
			\textbf{GitHub Actions}
			\begin{lstlisting}[language=yaml, style=Shell]
name: Test
on: [push]
jobs:
  test:
    runs-on: ubuntu-latest
    strategy:
      matrix:
        python: [3.8, 3.9, 3.10]
    steps:
      - uses: actions/checkout@v3
      - uses: actions/setup-python@v4
        with:
          python-version: ${{ matrix.python }}
      - run: pip install -r requirements.txt
      - run: pytest
			\end{lstlisting}
		\end{column}
		\begin{column}{0.5\textwidth}
			\textbf{GitLab CI}
			\begin{lstlisting}[language=yaml, style=Shell]
stages:
  - test

test:
  stage: test
  image: python:3.10
  parallel:
    matrix:
      - PYTHON_VERSION: ["3.8", "3.9", "3.10"]
  before_script:
    - pip install -r requirements.txt
  script:
    - pytest
			\end{lstlisting}
		\end{column}
	\end{columns}
\end{frame}

%%%%%%%%%%%%%%%%%%%%%%%%%%%%%%%%%%%%%%%%%%%%%%%%%%%%%%%%%%%%%%%%%%%%%%%%%%%%%%%%
\begin{frame}[fragile]
	\frametitle{GitLab Runner on HPC Systems}
	\begin{block}{Why HPC Runners?}
		{HPC systems provide powerful compute resources for CI/CD pipelines.}
	\end{block}
	\pause
	\textbf{Setup Overview:}
	\begin{enumerate}
		\item Install GitLab Runner on login/head node
		\item Configure runner to submit jobs to scheduler (SLURM/LSF)
		\item Set up custom executor or shell executor
		\item Configure resource limits and queues
	\end{enumerate}
	\pause
	\begin{lstlisting}[language=bash, style=Shell]
# Install GitLab Runner
curl -L --output /usr/local/bin/gitlab-runner \
  https://gitlab-runner-downloads.s3.amazonaws.com/latest/binaries/gitlab-runner-linux-amd64
chmod +x /usr/local/bin/gitlab-runner

# Register runner with custom executor
gitlab-runner register --executor custom
	\end{lstlisting}
\end{frame}

%%%%%%%%%%%%%%%%%%%%%%%%%%%%%%%%%%%%%%%%%%%%%%%%%%%%%%%%%%%%%%%%%%%%%%%%%%%%%%%%
\begin{frame}[fragile]
	\frametitle{HPC Runner Configuration}
	\textbf{Key considerations:}
	\begin{itemize}
		\item \textbf{Scheduler Integration:} Use \texttt{sbatch}/\texttt{qsub} in scripts
		\item \textbf{Resource Requests:} Configure CPU, memory, time limits
		\item \textbf{Queue Selection:} Route jobs to appropriate partitions
		\item \textbf{Environment:} Load modules, set paths in job scripts
	\end{itemize}
	\pause
	\textbf{Example SLURM integration:}
	\begin{lstlisting}[language=yaml, style=Shell]
test-hpc:
  stage: test
  tags: [hpc-runner]
  script:
    - |
      sbatch --wait --job-name=mpi-ci-test \
             --nodes=2 --ntasks-per-node=128 --time=30:00 \
             --wrap="module load mpi/openmpi && mpirun ./my_mpi_program"
	\end{lstlisting}
	\pause
	\begin{warning}[Resource Management]
		{Always set appropriate resource limits to avoid monopolizing HPC resources!}
	\end{warning}
\end{frame}

