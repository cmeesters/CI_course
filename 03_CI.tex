%%%%%%%%%%%%%%%%%%%%%%%%%%%%%%%%%%%%%%%%%%%%%%%%%%%%%%%%%%%%%%%%%%%%%%%%%%%%%%%%
\section{Going CI}
{   
	\usebackgroundtemplate{
		\vbox to \paperheight{\vfil\hbox to \paperwidth{\hfil\includegraphics[height=\paperheight]{pipe.png}\hfil}\vfil}
		%https://www.flaticon.com/free-icon/decoration_2788716
		%<a href="https://www.flaticon.com/free-icons/decoration" title="decoration icons">Decoration icons created by Freepik - Flaticon</a>
		
	}
	\frame{
		\frametitle{Let's move to a CI Pipleline}
		\begin{mdframed}[tikzsetting={draw=white,fill=white,fill opacity=0.8,
				line width=0pt},backgroundcolor=none,leftmargin=0,
			rightmargin=150,innertopmargin=4pt,roundcorner=10pt]
			\tableofcontents[currentsection,sections={1-6},hideothersubsections]
		\end{mdframed}
	}
}

%%%%%%%%%%%%%%%%%%%%%%%%%%%%%%%%%%%%%%%%%%%%%%%%%%%%%%%%%%%%%%%%%%%%%%%%%%%%%%%%
\begin{frame}[fragile]
   \frametitle{Preparations}
	All GitHub workflows are stored in a hidden directory \altverb{.github/workflows/}.
	\begin{task}
		{Create this directory.}
	\end{task}
    \begin{lstlisting}[language=Bash, style=Shell]
$ mkdir -p .github/workflows/
    \end{lstlisting}
    \pause
    \begin{docs}
    	{In GitLab all you need is a file \altverb{.gitlab-ci.yml}.}
    \end{docs}
\end{frame}

%%%%%%%%%%%%%%%%%%%%%%%%%%%%%%%%%%%%%%%%%%%%%%%%%%%%%%%%%%%%%%%%%%%%%%%%%%%%%%%%
\begin{frame}<handout:0>
	\begin{hint}
		{We are \emph{not} going to code all what follows by hand!}
	\end{hint}
\end{frame}

%%%%%%%%%%%%%%%%%%%%%%%%%%%%%%%%%%%%%%%%%%%%%%%%%%%%%%%%%%%%%%%%%%%%%%%%%%%%%%%%
\begin{frame}[fragile]
	\frametitle{Introducing the YAML files}
	Within the workflow directory, you may place yaml files.
	\begin{docs}
		{The naming is arbitrary!}
	\end{docs}
    First, we baptize our files for purpose, e.\,g. \altverb{CI}, \altverb{release-please}, etc.:
    \begin{lstlisting}[language=yaml,basicstyle=\small\ttfamily]
name: CI
    \end{lstlisting}
\end{frame}

%%%%%%%%%%%%%%%%%%%%%%%%%%%%%%%%%%%%%%%%%%%%%%%%%%%%%%%%%%%%%%%%%%%%%%%%%%%%%%%%
\begin{frame}[fragile]
	\frametitle{The \texttt{on} Property}
	\begin{columns}
		\begin{column}{0.5\textwidth}
			\begin{lstlisting}[language=yaml,basicstyle=\small\ttfamily]
on:
  push:
    branches:
      - main
  pull_request:
			\end{lstlisting}
		\end{column}
		\begin{column}{0.5\textwidth}
			Defines when the workflow should run:
			\begin{itemize}
				\item On pushes to the \texttt{main} branch
				\item On pull requests
			\end{itemize}
		\end{column}
	\end{columns}
\end{frame}

%%%%%%%%%%%%%%%%%%%%%%%%%%%%%%%%%%%%%%%%%%%%%%%%%%%%%%%%%%%%%%%%%%%%%%%%%%%%%%%%
\begin{frame}[fragile]
	\frametitle{The \texttt{jobs} Property}
	\begin{columns}
		\begin{column}{0.5\textwidth}
			\begin{lstlisting}[language=yaml,basicstyle=\small\ttfamily]
jobs:
  formatting:
    runs-on: ubuntu-latest
  linting:
    runs-on: ubuntu-latest
  testing:
    runs-on: ubuntu-latest
			\end{lstlisting}
		\end{column}
		\begin{column}{0.5\textwidth}
			Contains all jobs that run in the workflow:
			\begin{itemize}
				\item \texttt{formatting} - Code formatting checks
				\item \texttt{linting} - Code quality checks
				\item \texttt{testing} - Running tests
			\end{itemize}
		\end{column}
	\end{columns}
\end{frame}

%%%%%%%%%%%%%%%%%%%%%%%%%%%%%%%%%%%%%%%%%%%%%%%%%%%%%%%%%%%%%%%%%%%%%%%%%%%%%%%%
\begin{frame}[fragile]
	\frametitle{The \texttt{runs-on} Property}
	\begin{columns}
		\begin{column}{0.5\textwidth}
			\begin{lstlisting}[language=yaml,basicstyle=\small\ttfamily]
jobs:
  formatting:
    runs-on: ubuntu-latest
			\end{lstlisting}
		\end{column}
		\begin{column}{0.5\textwidth}
			Specifies the type of machine to run the job on:
			\begin{itemize}
				\item \texttt{ubuntu-latest} - Latest Ubuntu runner
				\item Other options: \texttt{windows-latest}, \texttt{macos-latest}
			\end{itemize}
		\end{column}
	\end{columns}
\end{frame}

%%%%%%%%%%%%%%%%%%%%%%%%%%%%%%%%%%%%%%%%%%%%%%%%%%%%%%%%%%%%%%%%%%%%%%%%%%%%%%%%
\begin{frame}[fragile]
	\frametitle{The \texttt{services} Property}
	\begin{columns}
		\begin{column}{0.5\textwidth}
			\begin{lstlisting}[language=yaml,basicstyle=\small\ttfamily]
services:
  mysql:
    image: mysql:8.0
    env:
      MYSQL_ROOT_PASSWORD: root
    ports:
      - "8888:3306"
			\end{lstlisting}
		\end{column}
		\begin{column}{0.5\textwidth}
			Provides additional services for testing:
			\begin{itemize}
				\item Runs MySQL database container
				\item Sets environment variables
				\item Maps ports for access
			\end{itemize}
		\end{column}
	\end{columns}
\end{frame}

%%%%%%%%%%%%%%%%%%%%%%%%%%%%%%%%%%%%%%%%%%%%%%%%%%%%%%%%%%%%%%%%%%%%%%%%%%%%%%%%
\begin{frame}[fragile]
	\frametitle{The \texttt{steps} Property}
	\begin{columns}
		\begin{column}{0.5\textwidth}
			\begin{lstlisting}[language=yaml,basicstyle=\small\ttfamily]
steps:
  - name: Check out the code
    uses: actions/checkout@v4
  - name: Install poetry
    run: pip install poetry
			\end{lstlisting}
		\end{column}
		\begin{column}{0.5\textwidth}
			Defines the sequence of tasks to execute:
			\begin{itemize}
				\item Each step can use pre-built actions (\texttt{uses})
				\item Or run shell commands (\texttt{run})
			\end{itemize}
		\end{column}
	\end{columns}
\end{frame}

%%%%%%%%%%%%%%%%%%%%%%%%%%%%%%%%%%%%%%%%%%%%%%%%%%%%%%%%%%%%%%%%%%%%%%%%%%%%%%%%
\begin{frame}[fragile]
	\frametitle{Using $3^\mathsf{rd}$ Party CI Actions}
	You probably noticed \altverb{uses: actions/checkout@v4}.\\
	The \altverb{uses} directives allows to use $3^\mathsf{rd}$ code for CI actions. It
	\begin{itemize}[<+->]
		\item eases CI pipeline development,
		\item will download the desired actions during runtime,
		\item may pose a security risk,
		\item will be covered later.
	\end{itemize}
\end{frame}

%%%%%%%%%%%%%%%%%%%%%%%%%%%%%%%%%%%%%%%%%%%%%%%%%%%%%%%%%%%%%%%%%%%%%%%%%%%%%%%%
\begin{frame}[fragile]
	\frametitle{The \texttt{uses} Property}
	\begin{columns}
		\begin{column}{0.5\textwidth}
			\begin{lstlisting}[language=yaml,basicstyle=\small\ttfamily]
- name: Check out the code
  uses: actions/checkout@v4
- uses: actions/setup-python@v5
  with:
    python-version: "3.11"
			\end{lstlisting}
		\end{column}
		\begin{column}{0.5\textwidth}
			References pre-built GitHub Actions:
			\begin{itemize}
				\item \texttt{actions/checkout} - Downloads repository code
				\item \texttt{actions/setup-python} - Sets up Python environment
				\item Version pinning with \texttt{@v4}, \texttt{@v5}
			\end{itemize}
		\end{column}
	\end{columns}
\end{frame}

%%%%%%%%%%%%%%%%%%%%%%%%%%%%%%%%%%%%%%%%%%%%%%%%%%%%%%%%%%%%%%%%%%%%%%%%%%%%%%%%
\begin{frame}[fragile]
	\frametitle{The \texttt{with} Property}
	\begin{columns}
		\begin{column}{0.5\textwidth}
			\begin{lstlisting}[language=yaml,basicstyle=\small\ttfamily]
- uses: actions/setup-python@v5
  with:
    python-version: "3.11"
    cache: poetry
			\end{lstlisting}
		\end{column}
		\begin{column}{0.5\textwidth}
			Provides input parameters to actions:
			\begin{itemize}
				\item \texttt{python-version} - Specifies Python version
				\item \texttt{cache} - Enables dependency caching
			\end{itemize}
		\end{column}
	\end{columns}
\end{frame}

%%%%%%%%%%%%%%%%%%%%%%%%%%%%%%%%%%%%%%%%%%%%%%%%%%%%%%%%%%%%%%%%%%%%%%%%%%%%%%%%
\begin{frame}[fragile]
	\frametitle{The \texttt{run} Property}
	\begin{columns}
		\begin{column}{0.5\textwidth}
			\begin{lstlisting}[language=yaml,basicstyle=\small\ttfamily]
- name: Install poetry
  run: pip install poetry
- name: Check formatting
  run: poetry run black --check .
			\end{lstlisting}
		\end{column}
		\begin{column}{0.5\textwidth}
			Executes shell commands:
			\begin{itemize}
				\item Single commands or multi-line scripts
				\item Uses the default shell of the runner
				\item Can include complex command sequences
			\end{itemize}
		\end{column}
	\end{columns}
\end{frame}

%%%%%%%%%%%%%%%%%%%%%%%%%%%%%%%%%%%%%%%%%%%%%%%%%%%%%%%%%%%%%%%%%%%%%%%%%%%%%%%%
\begin{frame}[fragile]
	\frametitle{\HandsOn{Dowload and Run}}
	\begin{task}
		{Please download the gist \url{https://jgu.to/w0bk}} as \altverb{ci.yml} in your \altverb{.github/workflows} directory.
	\end{task}
    Place it in \altverb{.github/workflows} and
    \begin{lstlisting}[language=Bash, style=Shell]
$ git add .github/workflows/ci.yml
$ git commit -m "your commit message"
$ git push
    \end{lstlisting}
    Then, look on the running actions of your forked project.
\end{frame}

%%%%%%%%%%%%%%%%%%%%%%%%%%%%%%%%%%%%%%%%%%%%%%%%%%%%%%%%%%%%%%%%%%%%%%%%%%%%%%%%
\begin{frame}[fragile]
	\frametitle{Looking at the Details}
	\begin{task}
		{Whilst the CI is running (a few minutes), we shall read the pipeline in detail. Ask, what you want to ask!}
	\end{task}
\end{frame}

%%%%%%%%%%%%%%%%%%%%%%%%%%%%%%%%%%%%%%%%%%%%%%%%%%%%%%%%%%%%%%%%%%%%%%%%%%%%%%%%
\begin{frame}[fragile]
	\frametitle{Correcting our Code}
	We noticed 3 failed jobs:
	\begin{itemize}[<+->]
		\item to correct \altverb{formatting} and \altverb{linting} we need to run
		      \begin{lstlisting}[language=Bash, style=Shell]
$ black @.@
$ #then
$ git add src/converter.py tests/test_converter.py \
> scripts/build.py
$ git commit -m "your commit message"
$ git push
		      \end{lstlisting}
	     \item for the "numerics" remember: $\circ\mathsf{C} = \frac{\circ\mathsf{F} - 32}{1.8}$ - do you find the spot?
	\end{itemize}
\end{frame}


%%%%%%%%%%%%%%%%%%%%%%%%%%%%%%%%%%%%%%%%%%%%%%%%%%%%%%%%%%%%%%%%%%%%%%%%%%%%%%%%
\begin{frame}[fragile]
	\frametitle{\HandsOn{Adding a \CC Pipline} }
	\begin{task}
		{Please download the gist \url{https://jgu.to/lnkz}}
	\end{task}
	Place it in \altverb{.github/workflows} and
	\begin{lstlisting}[language=Bash, style=Shell]
$ git add .github/workflows/ci.yml
$ git commit -m "your commit message"
$ git push
	\end{lstlisting}
	Then, look on the running actions of your forked project.
\end{frame}

%%%%%%%%%%%%%%%%%%%%%%%%%%%%%%%%%%%%%%%%%%%%%%%%%%%%%%%%%%%%%%%%%%%%%%%%%%%%%%%%
\begin{frame}[fragile]
	\frametitle{Looking at the Details}
	\begin{task}
		{Whilst the CI is running (a few minutes), we shall read the pipeline in detail. Ask, what you want to ask!}
	\end{task}
\end{frame}

%%%%%%%%%%%%%%%%%%%%%%%%%%%%%%%%%%%%%%%%%%%%%%%%%%%%%%%%%%%%%%%%%%%%%%%%%%%%%%%%
\begin{frame}[fragile]
	\frametitle{Correcting our Code}
	Here any number of jobs may fail - the first failing will prevent other runners from starting. We only need to correct the "numerics", then:
	\begin{lstlisting}[language=Bash, style=Shell]
$ git add 
$ git commit -m "your commit message"
$ git push
	\end{lstlisting}
\end{frame}